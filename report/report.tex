\documentclass{article}
\usepackage[margin=2cm]{geometry}
\usepackage[pdftex]{graphicx}
\begin{document}
\title{Learning to Play Atari Games}
\author{Matthew Hausknecht and Piyush Khandelwal}

\maketitle

\begin{abstract}
In this work we apply HyperNeat to the problem of learning how to play Atari games. By leveraging the geometric regularities present in the Atari game screen, HyperNeat is able to effectively evolve policies for playing several different Atari games. Results show that Hyperneat outperforms related RL and search techniques.
\end{abstract}

\section{Introduction}
Games have long been considered a fruitful domain for the study of AI. Seminal work on game playing includes Samuel's checkers playing program\cite{samuel_59} and Tesauro's TD-Gammon\cite{tesauro_94}. Games represent problems challenging enough to interest people yet abstract enough to be captured and modeled inside of computer programs. 

Common amoung many games is a representation of physical space. Board games as well as Atari games generally employ a 2-D overhead representation. More sophisticated computer games (as well as a few Atari games) use 3-D

\section{Background}

\section{HyperNeat}

\section{Results}

\section{Related Work}

\section{Conclusion}


\bibliographystyle{plain}	
\bibliography{report}
\end{document}

%% \begin{figure}[htp]
%% \begin{center}
%% \begin{array}{ccc}
%% \includegraphics[scale=.5]{part1a1.png} & \includegraphics[scale=.5]{part1a2.png} & \includegraphics[scale=.5]{part1a3.png} \\
%% \end{array}
%% \end{center}
%% \caption{Uni-dimensional self organizing maps consisting of 1000, 2000 and 3000 units.}
%% \label{fig1}
%% \end{figure}








